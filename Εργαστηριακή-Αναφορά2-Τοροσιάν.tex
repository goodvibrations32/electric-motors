% Created 2023-02-23 Πεμ 00:22
% Intended LaTeX compiler: pdflatex
\documentclass[11pt]{article}
\usepackage[utf8]{inputenc}
\usepackage[T1]{fontenc}
\usepackage{graphicx}
\usepackage{longtable}
\usepackage{wrapfig}
\usepackage{rotating}
\usepackage[normalem]{ulem}
\usepackage{amsmath}
\usepackage{amssymb}
\usepackage{capt-of}
\usepackage{hyperref}
\usepackage{minted}
\usepackage[margin=2cm]{geometry}
\usepackage{setspace}
\usepackage[utf8]{inputenc}
\usepackage[LGR]{fontenc}
\usepackage[greek, greek, ]{babel}
\usepackage[T1]{fontenc}
\usepackage[english,greek]{babel}
\newcommand{\en}[1]{\foreignlanguage{english}{#1}}
\usepackage{minted}
\usepackage[hidelinks]{hyperref}
\hypersetup{colorlinks=true, linkcolor=black}
\author{Τοροσιάν Νικόλαος ΤΜ6220}
\date{\today}
\title{Τριφασικοί μετασχηματιστές\\\medskip
\large Εργαστηριακή άσκηση 2}
\hypersetup{
 pdfauthor={Τοροσιάν Νικόλαος ΤΜ6220},
 pdftitle={Τριφασικοί μετασχηματιστές},
 pdfkeywords={},
 pdfsubject={},
 pdfcreator={Emacs 28.2 (Org mode 9.6)}, 
 pdflang={Gr}}
\begin{document}

\maketitle
\section{Μετρήσεις εργαστηρίου}
\label{sec:org996528c}
\subsection{Αστέρας-Αστέρας}
\label{sec:orgcd5a3cb}
\begin{center}
\begin{tabular}{rrrr}
\en{V1 (V)} & \en{V2 (V)} & \en{I (A)} & \en{P (W)}\\\empty
\hline
230 & 228.4 & 0.182 & 75.5\\\empty
170 & 168.8 & 0.12 & 46.5\\\empty
\end{tabular}
\end{center}
\subsection{Αστέρας-Τρίγωνο}
\label{sec:org124ba79}
\begin{center}
\begin{tabular}{rrrr}
\en{V1 (V)} & \en{V2 (V)} & \en{I (A)} & \en{P (W)}\\\empty
\hline
230 & 228.4 & 0.182 & 75.5\\\empty
170 & 168.8 & 0.12 & 46.5\\\empty
\end{tabular}
\end{center}

\subsection{Αστέρας-Αστέρας με φορτίο}
\label{sec:org086d544}
\begin{center}
\begin{tabular}{rrrrrr}
\en{P1 (W)} & \en{P2 (W)} & \en{V1 (V)} & \en{V2 (V)} & \en{I1 (A)} & \en{I2 (A)}\\\empty
\hline
648 & 620 & 228 & 295 & 2.875 & 2.7\\\empty
860 & 830 & 229 & 297 & 3.8 & 3.71\\\empty
\end{tabular}
\end{center}

\begin{itemize}
\item Κύκλωμα \en{RC}
\end{itemize}
\begin{center}
\begin{tabular}{rrrrrr}
\en{P1 (W)} & \en{P2 (W)} & \en{V1 (V)} & \en{V2 (V)} & \en{I1 (A)} & \en{I2 (A)}\\\empty
\hline
700 & 640 & 230 & 305 & 6.1 & 6.22\\\empty
786 & 645 & 231 & 306 & 9.78 & 9.91\\\empty
844 & 660 & 231 & 307 & 14.6 & 14.7\\\empty
\end{tabular}
\end{center}
\section{Να υπολογιστούν οι ηλεκτρικές απώλειες ισχύος για κάθε κατάσταση}
\label{sec:orgb86675a}

\begin{itemize}
\item Αστέρας-Αστέρας με φορτίο
\begin{equation}
\begin{align}
W = P_{1}-P_{2} \Rightarrow \\
W_{1} &= 648-620 [W] = 28 [W] \\
W_{2} &= 860-830 [W] = 30 [W] \\
RC \\
W_{1} &= 700-640 [W] = 60 [W] \\
W_{2} &= 786-645 [W] = 141[W] \\
W_{3} &= 844-660 [W] = 184[W] \\
\end{align}
\end{equation}
\end{itemize}

\section{Να υπολογιστεί ο βαθμός απόδοσης του μετασχηματιστή}
\label{sec:org05a1aa6}

\begin{equation}
\begin{align}
n = \frac{P_{2}}{P_{1}} \cdot 100 \% \Rightarrow \\
n_{1} &=\frac{620}{648} [\%] = 95.6 \%  \\
n_{2} &=\frac{830}{860} [\%] = 96.5 \% \\
RC \\
n_{1} &=\frac{640}{700} [\%] = 91.4 \% \\
n_{2} &=\frac{645}{786} [\%] = 82.1\% \\
n_{3} &=\frac{660}{844} [\%] = 78.2\% \\
\end{align}
\end{equation}

\section{Να υπολογιστεί ο συντελεστής ισχύος για κάθε φόρτιση}
\label{sec:orgb72dfb9}
\begin{equation}
\begin{align}
\cos(\phi)=\frac{P}{S} = \frac{P}{V \cdot I} \Rightarrow \\
\cos(\phi_{1})=\frac{P_{1}}{V_{1} \cdot I_{1}} = \frac{648}{228 \cdot 2.875} &= 0.9885 \\
\cos(\phi_{1})= \frac{860}{229 \cdot 3.8} &= 0.9885 \\
\cos(\phi_{2})=\frac{P_{2}}{V_{2} \cdot I_{2}} = \frac{620}{295 \cdot 2.7} &= 0.778 \\
\cos(\phi_{2})= \frac{830}{297 \cdot 3.71} &= 0.753 \\
\end{align}
\end{equation}

\subsection{Συντελεστής ισχύος για Ωμικό και χωριτικό φορτίο}
\label{sec:org987feeb}
\begin{equation}
\begin{align}
cos(\phi_{1})=\frac{P_{1}}{V_{1} \cdot I_{1}} = \frac{700}{230 \cdot 6.1} &= 0.499 \\
cos(\phi_{1})= \frac{786}{231 \cdot 9.78} &= 0.378 \\
cos(\phi_{1})= \frac{844}{231 \cdot 14.6} &= 0.25 \\
cos(\phi_{2})=\frac{P_{2}}{V_{2} \cdot I_{2}} = \frac{640}{305 \cdot 6.2} &= 0.337 \\
cos(\phi_{2})= \frac{645}{306 \cdot 9.91} &= 0.213 \\
cos(\phi_{2})= \frac{660}{307 \cdot 14.7} &= 0.146 \\
\end{align}
\end{equation}
\section{Τελικοί πίνακες και διαγράμματα}
\label{sec:org176f767}
\subsection{Αποτελέσματα}
\label{sec:org482377c}
Ανανεώνοντας τους ανωτέρω πίνακες μας δίνεται η δυνατότητα για την εξαγωγή των διαγραμμάτων του βαθμού απόδοσης του μετασχηματιστή, της ισχύος εξόδου (\en{W}) καθώς και της τάσης εξόδου (\en{V}), ως προς την ένταση του μετασχηματιστή.
\subsubsection{Αστέρας-Αστέρας με φορτίο}
\label{sec:org892b1ae}
\begin{table}[htbp]
\label{my-data}
\centering
\begin{tabular}{rrrrrrrr}
\en{P1 (W)} & \en{P2 (W)} & \en{V1 (V)} & \en{V2 (V)} & \en{I1 (A)} & \en{I2 (A)} & \en{n (\%)} & \en{P (W)}\\\empty
\hline
648 & 620 & 228 & 295 & 2.875 & 2.7 & 95.6 & 28\\\empty
860 & 830 & 229 & 297 & 3.8 & 3.71 & 96.5 & 30\\\empty
700 & 640 & 230 & 305 & 6.1 & 6.22 & 91.4 & 60\\\empty
786 & 645 & 231 & 306 & 9.78 & 9.91 & 82.1 & 141\\\empty
844 & 660 & 231 & 307 & 14.6 & 14.7 & 78.2 & 184\\\empty
\end{tabular}
\end{table}

\subsubsection{Διαγράμματα}
\label{sec:orgb191ee0}
Για την εξαγωγή των διαγραμμάτων χρησιμοποιήθηκε ένα πρόγραμμα γραμμής εντολών \en{(gnuplot)} που είναι διαθέσιμο σε όλα τα λειτουργικά συστήματα \en{Operating Systems (Windows, Mac, linux)} και παρέχουν πολλές δυνατότητες στον χρήστη. Παρακάτω φαίνεται ο κώδικας που χρησιμοποιήθηκε για την εξαγωγή των διαγραμμάτων αφού περαστούν οι απαραίτητοι πίνακες στο πρόγραμμα.

\begin{enumerate}
\item Διάγραμμα απόδοσης έντασης
\label{sec:org0472f3e}

\begin{center}
\includegraphics[width=.9\linewidth]{eff-amp.png}
\end{center}

\selectlanguage{english}
\begin{minted}[]{gnuplot}
# Grid
set style line 102 lc rgb'#808080' lt 0 lw 1
set grid back ls 102

set ylabel "n (%)"
set xlabel "Current [A]"
set title "efficiency in relation to current"

set xrange [0:16]
set xtics 0,1,16
set yrange [0:100]
set ytics 0,10,100
set key bottom right

plot data u 1:2 w lines lw 2 title "Transformer"
\end{minted}



\selectlanguage{greek}
\item Διάγραμμα Ισχύος Έντασης
\label{sec:org18b7590}

\begin{center}
\includegraphics[width=.9\linewidth]{power.png}
\end{center}

\selectlanguage{english}
\begin{minted}[]{gnuplot}
# Grid
set style line 102 lc rgb'#808080' lt 0 lw 1
set grid back ls 102

set ylabel "Power (W)"
set xlabel "Current [A]"
set title "Power in relation to Current"

set xrange [0:16]
set xtics 0,1,16
set yrange [0:200]
set ytics 0,20,200
set key bottom right

plot data u 1:3 w lines lw 2 title "Output power"
\end{minted}



\selectlanguage{greek}

\item Διάγραμμα Τάσης έντασης
\label{sec:org4862eb4}

\begin{center}
\includegraphics[width=.9\linewidth]{voltage.png}
\end{center}

\selectlanguage{english}
\begin{minted}[]{gnuplot}
# Grid
set style line 102 lc rgb'#808080' lt 0 lw 1
set grid back ls 102

set ylabel "Voltage (V)"
set xlabel "Current [A]"
set title "Voltage vs Current"

set xrange [0:16]
set xtics 0,1,16
set yrange [290:310]
set ytics 220,5,320
set key bottom right

plot data u 1:2 w lines lw 2 title "voltage curve"
\end{minted}

\selectlanguage{greek}
\end{enumerate}
\end{document}