% Created 2023-02-27 Δευ 17:03
\documentclass[11pt]{article}
\usepackage[utf8]{inputenc}
\usepackage[T1]{fontenc}
\usepackage{graphicx}
\usepackage{longtable}
\usepackage{wrapfig}
\usepackage{rotating}
\usepackage[normalem]{ulem}
\usepackage{amsmath}
\usepackage{amssymb}
\usepackage{capt-of}
\usepackage{hyperref}
\usepackage{minted}
\usepackage[margin=2cm]{geometry}
\usepackage{setspace}
\usepackage[utf8]{inputenc}
\usepackage[LGR]{fontenc}
\usepackage[greek, greek, ]{babel}
\usepackage[T1]{fontenc}
\usepackage[english,greek]{babel}
\newcommand{\en}[1]{\foreignlanguage{english}{#1}}
\usepackage{minted}
\usepackage[hidelinks]{hyperref}
\hypersetup{colorlinks=true, linkcolor=black}
\author{Τοροσιάν Νικόλαος ΤΜ6220}
\date{\today}
\title{Κινητήρας συνεχούς ρεύματος\\\medskip
\large Εργαστηριακή άσκηση 4}
\hypersetup{
 pdfauthor={Τοροσιάν Νικόλαος ΤΜ6220},
 pdftitle={Κινητήρας συνεχούς ρεύματος},
 pdfkeywords={},
 pdfsubject={},
 pdfcreator={Emacs 28.2 (Org mode 9.6)}, 
 pdflang={Gr}}
\begin{document}

\maketitle
\section{Μετρήσεις εργαστηρίου}
\label{sec:orge9e9ff5}
\subsection{Κινητήρας και γεννήτρια παράλληλης διέγερσης}
\label{sec:org5f68716}
\subsubsection{Εν κενώ}
\label{sec:orgdaf39d6}
\begin{table}[htbp]
\label{parallelTrans}
\centering
\begin{tabular}{rrr}
\(\en{n (rpm)}\) & \(\en{I_{f} (A)}\) & \(\en{V_{t} (V)}\)\\\empty
\hline
262 & 0.058 & 27.2\\\empty
961 & 0.085 & 39\\\empty
1276 & 0.11 & 54\\\empty
1460 & 0.145 & 67.7\\\empty
1553 & 0.17 & 80.5\\\empty
1685 & 0.19 & 93.4\\\empty
1873 & 0.227 & 107.8\\\empty
1880 & 0.252 & 120\\\empty
1896 & 0.265 & 128.9\\\empty
\end{tabular}
\end{table}
\subsubsection{Υπό φορτίο}
\label{sec:org4c5253f}
Για τους υπολογισμούς της επαγωγικής τάσης και της στρεπτικής ροπής εξόδου του κινητήρα καθώς και για την γωνιακή ταχύτητά \(\omega\) του, χρησιμοποιήθηκαν οι παρακάτω σχέσεις:

\begin{equation}
\begin{align}
R_{T} = 8.3 \Omega \\
R_{F} = 459 \Omega \\
\omega = 2 \cdot \pi \cdot \frac{n}{60} \\
&I_{t} = I_{l} + \frac{V_{t}}{R_{F}} \\
&U_{ep} = V_{t} - I_{t} \cdot R_{T} \\
&T = \frac{U_{ep} \cdot I_{t} \cdot 9.55}{\omega} \\
\end{align}
\end{equation}


Παρακάτω φαίνονται οι πίνακες με συμπληρωμένα τα στοιχεία των πράξεων για κάθε μέτρηση
\begin{enumerate}
\item Κινητήρας : Εδώ παρατηρούμε πως η στρεπτική ροπή του κινητήρα μειώνεται όσο αυξάνουν οι
\label{sec:org2b40469}
στροφές, πράγμα που μας επιβεβαιώνει πως οι μετρήσεις και οι πράξεις έγιναν
σωστά.
\begin{table}[htbp]
\caption{\label{onload}Κινητήρας}
\centering
\begin{tabular}{rrrrrrr}
\(\en{I_{f}}\) & \(\en{V_{t}}\) & \(\en{I_{l}}\) & \(\en{rpm}\) & \(\en{U_{ep}}\) & \(\en{I_{t}}\) & \(\en{T}\)\\\empty
\hline
0.16 & 79.6 & 3.15 & 1290 & 52.0 & 3.32 & 12.2\\\empty
0.16 & 79.3 & 2.92 & 1357 & 53.7 & 3.09 & 11.2\\\empty
0.163 & 78.9 & 3.01 & 1362 & 52.5 & 3.18 & 11.2\\\empty
0.16 & 79.1 & 2.9 & 1370 & 53.6 & 3.07 & 11.0\\\empty
0.16 & 80.4 & 2.59 & 1420 & 57.4 & 2.77 & 10.2\\\empty
0.169 & 82 & 2.071 & 1553 & 63.3 & 2.25 & 8.36\\\empty
0.171 & 83.2 & 1.69 & 1640 & 67.7 & 1.87 & 7.04\\\empty
\end{tabular}
\end{table}
\item Γεννήτρια : Αντίστοιχα εδώ φαίνονται οι τιμές του ρεύματος και της τάσης της γεννήτριας για την ενλόγω διάταξη.
\label{sec:orga9244a5}
\begin{table}[htbp]
\caption{\label{rand}Γεννήτρια}
\centering
\begin{tabular}{rr}
\(\en{I (A)}\) & \(\en{V(V)}\)\\\empty
\hline
0.75 & 88\\\empty
0.86 & 55\\\empty
1.1 & 45\\\empty
1.2 & 39\\\empty
1.38 & 30\\\empty
1.43 & 20\\\empty
1.45 & 11\\\empty
\end{tabular}
\end{table}
\end{enumerate}
\subsection{Κινητήρας ξένης διέγερσης και γεννήτρια παράλληλης διέγερσης}
\label{sec:org1e4d1c1}
\subsubsection{Μεταβολή τάσης \(\en{V_{t}}\) και ρεύματος \(\en{I_{f}}\)}
\label{sec:org55ab2ae}
\begin{table}[htbp]
\caption{\label{voltage}Μεταβολή τάσης}
\centering
\begin{tabular}{rrr}
\(\en{I_{f}}\) & \(\en{V_{t}}\) & \(\en{n}\)\\\empty
\hline
0.33 & 18.8 & 223\\\empty
0.33 & 31.6 & 413\\\empty
0.33 & 43.8 & 597\\\empty
0.33 & 58.6 & 820\\\empty
0.33 & 72.7 & 1033\\\empty
0.33 & 85.4 & 1222\\\empty
0.33 & 97.2 & 1374\\\empty
0.33 & 109.2 & 1511\\\empty
0.33 & 121.5 & 1652\\\empty
0.33 & 130.2 & 1750\\\empty
\end{tabular}
\end{table}

\begin{table}[htbp]
\caption{\label{current}Μεταβολή έντασης}
\centering
\begin{tabular}{rrr}
\(\en{V_{t}}\) & \(\en{I_{f}}\) & \(\en{n}\)\\\empty
\hline
70 & 0.323 & 1000\\\empty
70 & 0.286 & 1080\\\empty
70 & 0.248 & 1183\\\empty
70 & 0.211 & 1295\\\empty
70 & 0.17 & 1394\\\empty
70 & 0.14 & 1440\\\empty
70 & 0.11 & 1333\\\empty
70 & 0.097 & 1200\\\empty
\end{tabular}
\end{table}
\subsection{Κινητήρας διέγερσης σε σειρά και γεννήτρια παράλληλης διέγερσης}
\label{sec:orga222533}
\subsubsection{Υπό φορτίο}
\label{sec:orgff6cee4}
Για τους υπολογισμούς της επαγωγικής τάσης και της στρεπτικής ροπής εξόδου του κινητήρα καθώς και της γωνιακής ταχύτητας \(\omega\), χρησιμοποιήθηκαν οι παρακάτω σχέσεις:
\begin{equation}
\begin{align}
R_{T} = 8.3 \Omega \\
R_{F} = 1.5 \Omega \\
\omega = 2 \cdot \pi \cdot \frac{n}{60} \\
&U_{ep} = V_{t} - I_{t} \cdot (R_{T}+R_{F}) \\
&T = \frac{U_{ep} \cdot I_{t} \cdot 9.55}{\omega} \\
\end{align}
\end{equation}
\begin{table}[htbp]
\caption{\label{series}Κινητήρας σε σειρά και γεννήτρια παράλληλα συνδεδεμένη υπό φορτίο}
\centering
\begin{tabular}{rrrrrr}
\(\en{It}\) & \(\en{Vt}\) & \(\en{rpm}\) & \(\omega\) & \(\en{Uep}\) & \(\en{T}\)\\\empty
\hline
4.23 & 91.6 & 1350 & 141.4 & 50.15 & 14.33\\\empty
4.17 & 91.7 & 1359 & 142.3 & 50.83 & 14.23\\\empty
4.23 & 91.2 & 1300 & 136.1 & 49.75 & 14.77\\\empty
4.15 & 91.5 & 1340 & 140.3 & 50.83 & 14.36\\\empty
4.06 & 91.7 & 1371 & 143.6 & 51.91 & 14.02\\\empty
3.86 & 92.7 & 1510 & 158.1 & 54.87 & 12.79\\\empty
3.5 & 94.2 & 1640 & 171.7 & 59.9 & 11.66\\\empty
3.04 & 96.2 & 1810 & 189.5 & 66.41 & 10.17\\\empty
2.6 & 98.4 & 2160 & 226.2 & 72.92 & 8.004\\\empty
\end{tabular}
\end{table}


\twocolumn
Εδώ φαίνεται πάλι πως η ροπή και οι στροφές έχουν αντιστρόφως ανάλογη σχέση και πως η αύξηση των στροφών συνεπάγεται πτώση της έντασης ρεύματος του τυμπάνου καθώς και αύξηση της επαγωγικής τάσης.
Παρατηρούμε στην έναρξη του πειράματος μικρές διακυμάνσεις στις τιμές των στροφών του κινητήρα όπως και στην τάση και την ένταση του ρεύματος του τυμπάνου. Υποθέτω πως αυτό συνέβη λόγω της παλαιότητας του εξοπλισμού, αφού μετά την σταθεροποίηση του συστήματος περνώντας τις 1400 στροφές ανά λεπτό καταγράφεται η επιθυμητή συμπεριφορά. Τελικά λόγω της εξαγωγής των αποτελεσμάτων, ροπή, επαγωγική τάση και στρεπτική ροπή, με βάση την εκάστοτε κατάσταση το σφάλμα δεν ακολουθεί τις επόμενες μετρήσεις και έτσι οι ταλαντωτικές συμπεριφορές δεν μεταφέρονται σε όλα τα αποτελέσματα.
\begin{table}[htbp]
\caption{\label{series}Μετρήσεις και στοιχεία υπολογισμών κινητήρα σε σειρά}
\centering
\begin{tabular}{rrrrrr}
\(\en{It}\) & \(\en{Vt}\) & \(\en{rpm}\) & \(\omega\) & \(\en{Uep}\) & \(\en{T}\)\\\empty
\hline
4.23 & 91.6 & 1350 & 141.4 & 50.15 & 14.33\\\empty
4.17 & 91.7 & 1359 & 142.3 & 50.83 & 14.23\\\empty
4.23 & 91.2 & 1300 & 136.1 & 49.75 & 14.77\\\empty
4.15 & 91.5 & 1340 & 140.3 & 50.83 & 14.36\\\empty
4.06 & 91.7 & 1371 & 143.6 & 51.91 & 14.02\\\empty
3.86 & 92.7 & 1510 & 158.1 & 54.87 & 12.79\\\empty
3.5 & 94.2 & 1640 & 171.7 & 59.9 & 11.66\\\empty
3.04 & 96.2 & 1810 & 189.5 & 66.41 & 10.17\\\empty
2.6 & 98.4 & 2160 & 226.2 & 72.92 & 8.004\\\empty
\end{tabular}
\end{table}

\begin{table}[htbp]
\caption{\label{ser-gen}Γεννήτρια σε σειρά}
\centering
\begin{tabular}{rr}
\(\en{I (A)}\) & \(\en{V(V)}\)\\\empty
\hline
0.97 & 92\\\empty
0.1 & 83\\\empty
1.27 & 71\\\empty
1.27 & 60\\\empty
1.44 & 49\\\empty
1.65 & 42\\\empty
1.88 & 30\\\empty
1.94 & 20\\\empty
1.6 & 10\\\empty
\end{tabular}
\end{table}

Στο πείραμα εν κενό παρατηρούμε πως οι τιμές των στροφών δεν παρουσιάζουν διακυμάνσεις κατά την διεξαγωγή του πειράματος. Αυτό συμβαίνει γιατί στο τελευταίο πείραμα ο κινητήρας ζορίζεται μάλλον περισσότερο να εκκινήσει την γεννήτρια σε σχέση με το πρώτο πείραμα. Σε όλα τα πειράματα παρατηρείται από τα διαγράμματα που θα παρατεθούν παρακάτω πως εμφανίζεται μια μικρή ασυνέχεια της διεγεγραμμένης εξίσωσης που ενώνει τα σημεία των μετρήσεων η οποία μπορεί να αγνοηθεί. Είναι πιθανό να οφείλεται στο μικρό πλήθος τιμών που καταγράφηκαν, δηλαδή σε στατιστικό σφάλμα ή μπορεί να είναι και προϊόν του προγράμματος επεξεργασίας των δεδομένων (\en{gnuplot}).
\begin{table}[htbp]
\caption{\label{onload}Κινητήρας παράλληλα συνδεδεμένος}
\centering
\begin{tabular}{rrrrrrr}
\(\en{I_{f}}\) & \(\en{V_{t}}\) & \(\en{I_{l}}\) & \(\en{rpm}\) & \(\en{U_{ep}}\) & \(\en{I_{t}}\) & \(\en{T}\)\\\empty
\hline
0.16 & 79.6 & 3.15 & 1290 & 52.0 & 3.32 & 12.2\\\empty
0.16 & 79.3 & 2.92 & 1357 & 53.7 & 3.09 & 11.2\\\empty
0.163 & 78.9 & 3.01 & 1362 & 52.5 & 3.18 & 11.2\\\empty
0.16 & 79.1 & 2.9 & 1370 & 53.6 & 3.07 & 11.0\\\empty
0.16 & 80.4 & 2.59 & 1420 & 57.4 & 2.77 & 10.2\\\empty
0.169 & 82 & 2.071 & 1553 & 63.3 & 2.25 & 8.36\\\empty
0.171 & 83.2 & 1.69 & 1640 & 67.7 & 1.87 & 7.04\\\empty
\end{tabular}
\end{table}
\begin{table}[htbp]
\caption{\label{rand}Γεννήτρια παράλληλα}
\centering
\begin{tabular}{rr}
\(\en{I (A)}\) & \(\en{V(V)}\)\\\empty
0.75 & 88\\\empty
\hline
0.86 & 55\\\empty
1.1 & 45\\\empty
1.2 & 39\\\empty
1.38 & 30\\\empty
1.43 & 20\\\empty
1.45 & 11\\\empty
\end{tabular}
\end{table}

\onecolumn
\section{Ερωτήσεις αναφοράς}
\label{sec:org0b76b2c}
\subsection{1η ερώτηση}
\label{sec:org12a7a59}
Αν παρέχουμε στο τύμπανο του κινητήρα την μέγιστη τάση για την εκκίνησή του, λόγω της μικρής απαιτούμενης έντασης και φορτίου για την εκκίνηση, θα είχαμε μεγάλες θερμικές απώλειες και υποβάθμιση της λειτουργίας του κινητήρα. Μια από τις λύσεις στο πρόβλημα αυτό είναι η εγκατάσταση τροφοδοτικού μεταβαλλόμενης τάσης εξόδου ή με την χρήση θυρίστορ για την σταδιακή αύξηση και αποφυγή των ανεπιθύμητων απωλειών.
\subsection{2η ερώτηση}
\label{sec:org7e3393c}
Σε περίπτωση διακοπής της ροής ρεύματος στο κύκλωμα διέγερσης του κινητήρα, \(\en{I_{f}=0}\), οι στροφές του κινητήρα τείνουν στο άπειρο πράγμα αφύσικο. Για να αποφευχθεί αυτό θα πρέπει να υπάρχει πρόβλεψη για εγκατάσταση συστήματος διακοπής της τροφοδοσίας του κινητήρα σε τέτοια περίπτωση.
\subsection{3η ερώτηση}
\label{sec:org69223fd}
Ο κινητήρας σειράς έχει ως χαρακτηριστικό την ταχεία ανάπτυξη ροπής κατά την εκκίνησή του, ιδιότητα που τον καθιστά ιδανικό για συστήματα κίνησης οχημάτων, ανυψωτικά καθώς και συστήματα ρυμούλκησης. Επίσης οι στροφές του κινητήρα μπορούν να ρυθμιστούν με την τοποθέτηση αντίστασης παράλληλα στο τύλιγμα διεγέρσεως.
\section{Διαγράμματα}
\label{sec:orga5a11c6}
\begin{center}
\includegraphics[width=.9\linewidth]{mot-for-gen-par.png}
\end{center}

\begin{center}
\includegraphics[width=.9\linewidth]{rot-mot-ser-gen-par.png}
\end{center}

\begin{center}
\includegraphics[width=.9\linewidth]{tor-mot-ser-gen-par.png}
\end{center}

\begin{center}
\includegraphics[width=.9\linewidth]{volt-par.png}
\end{center}

\begin{center}
\includegraphics[width=.9\linewidth]{current-par.png}
\end{center}

\begin{center}
\includegraphics[width=.9\linewidth]{onload-rot-par.png}
\end{center}

\begin{center}
\includegraphics[width=.9\linewidth]{onload-strs-par.png}
\end{center}
\end{document}